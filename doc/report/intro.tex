Большинство программных систем, имеющих сложную структуру и состоящих из
нескольких сотен различных компонент, обладают рядом схожих проблем.
Например веб-поиск содержит следующие компоненты:
балансеры, верхние, средние метапоиски, промежуточные и базовые поиски,
колдунщики, антироботы, свежесть, региональные поиск,
несколько десятков параллельных поисков.
Краткое схематичное описание архитектуры:

\includegraphics[width=\textwidth]{pics/search.png}

Всего единовременно запущено несколько ******** тысяч инстансов
\footnote{инстанс --- приложение, запущенное в контейнере и описываемое парой host:port}
приложений. Каждый инстанс генерирует множество ошибок и записывает каждую
из них в log-файл.

Некоторые приложения, близкие по функционалу, пишут в один и тот же файл.
Log-файлы ротируются согласно определённому алгоритму. Тем не менее объем
log-файла для одного инстанса может достигать нескольких сотен мегабайт,
что препятствует быстрому ручному анализу в случае инцидента и инженеры
вынуждены тратить ценные секунды на просмотр сотен тысяч строк файла в поисках
сообщения с описанием элемента, вызвавшего сбой работы системы.

Начальным требованием к системе для эффективного использования алгоритма
является наличие сопоставимого с количеством поисковых приложений количества
нод на которых может быть запущена программа, реализующая алгоритм.

В организации, в которой выполнялась учебно-исследовательская работа,
развёрнута большая поисковая инфраструктура, которая не лишена недостатков и
существует вероятность поломки некоторой её части. Существует множество
средств мониторинга состояния веб-поиска и противодействия инцидентам,
но в некоторых случаях инженерам их недостаточно и приходится вручную
анализировать log-файлы отдельных инстансов приложений на отдельных host'ах,
что, в свою очередь, замедляет скорость реакции на непредвиденную ситуацию.
Но даже автоматизация процесса анализа log-файла на одного инстанса не решает
проблему полностью, поэтому необходима возможность быстрого анализа log-файлов
сразу множества инстансов.

Таким образом, целью этой учебно-исследовательской работы является разработка
алгоритма, позволяющего собирать статистику по ошибкам, встречающимся
в log-файлах инстансов поисковых приложений на основе существующих
шаблонов и выделять новые шаблоны.
