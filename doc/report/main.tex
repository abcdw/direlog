\documentclass[a4paper,12pt]{report}

\usepackage[unicode,colorlinks=true,linkcolor=blue]{hyperref}
% здесь подключении шрифтов в русскими буквами
\usepackage[T2A]{fontenc}
\usepackage[utf8]{inputenc}
\usepackage[english,russian]{babel}
\usepackage{amsmath,amsthm,amssymb,amsfonts,mathtext,cite,enumerate,float}
%\usepackage[dvips]{graphicx}
\usepackage[pdftex]{graphicx}
\graphicspath{{images/}}
\usepackage{fix-cm}

\makeatletter
\renewcommand{\@biblabel}[1]{#1.}
\makeatother

\usepackage{geometry}  % Меняем поля страницы
\geometry{left=2cm}    % левое поле
\geometry{right=1.5cm} % правое поле
\geometry{top=1cm}     % верхнее поле
\geometry{bottom=2cm}  % нижнее поле

\newcommand{\HRule}{\rule{\linewidth}{0.5mm}}

\renewcommand{\theenumi}{\arabic{enumi}}                                       % Меняем везде перечисления на цифра.цифра
\renewcommand{\labelenumi}{\arabic{enumi}}                                     % Меняем везде перечисления на цифра.цифра
\renewcommand{\theenumii}{.\arabic{enumii}}                                    % Меняем везде перечисления на цифра.цифра
\renewcommand{\labelenumii}{\arabic{enumi}.\arabic{enumii}.}                   % Меняем везде перечисления на цифра.цифра
\renewcommand{\theenumiii}{.\arabic{enumiii}}                                  % Меняем везде перечисления на цифра.цифра
\renewcommand{\labelenumiii}{\arabic{enumi}.\arabic{enumii}.\arabic{enumiii}.} % Меняем везде перечисления на цифра.цифра

\theoremstyle{plain}
\newtheorem{theorem}{Теорема}[section]
\newtheorem{lemma}[theorem]{Лемма}
\theoremstyle{definition}
\newtheorem{definition}{Определение}
\theoremstyle{remark}
\newtheorem*{example}{Пример}
\newtheorem*{remark}{Замечание}


\begin{document}
  % \begin{titlepage}
  \begin{center}

    МИНИСТЕРСТВО ОБРАЗОВАНИЯ И НАУКИ РОССИЙСКОЙ ФЕДЕРАЦИИ
    ФЕДЕРАЛЬНОЕ ГОСУДАРСТВЕННОЕ АВТОНОМНОЕ ОБРАЗОВАТЕЛЬНОЕ УЧРЕЖДЕНИЕ
    ВЫСШЕГО ПРОФЕССИОНАЛЬНОГО ОБРАЗОВАНИЯ \\[1em]
    Национальный исследовательский ядерный университет\\
    ``МИФИ'' \\[1em]

    \begin{minipage}{\textwidth}
      \begin{flushleft}
        \begin{tabular}{ l l }
          Факультет & Кибернетики и информационной безопасности\\
          Кафедра   & Информационные технологии в социальных системах
        \end{tabular}
      \end{flushleft}
    \end{minipage}\\[1em]

    \begin{minipage}{\textwidth}
      \begin{flushright}
        \textit{К защите допущен}\\
        Зам. заведующего кафедрой\\
        \underline{\hspace*{3.0cm}} /Сергиевский~М.\,В./\\
        ``\underline{\hspace*{1.5cm}}''
        \underline{\hspace*{4.0cm}}
        201\underline{\hspace*{0.5cm}}г.
      \end{flushright}
    \end{minipage}\\[3em]

    {ПОЯСНИТЕЛЬНАЯ ЗАПИСКА}\\
    {к учебно-исследовательской работе}\\
    {и курсовому проекту}\\
    {на тему:}\\[1em]
    \textbf{РАЗРАБОТКА АЛГОРИТМА ДЛЯ ВЫДЕЛЕНИЯ ЧАСТОВСТРЕЧАЮЩИХСЯ ШАБЛОНОВ
      ОШИБОК ИЗ ФАЙЛОВ ЖУРНАЛОВ ПРИЛОЖЕНИЙ}\\[1em]


    % {БГУИР ДП \textbf{1-31 03 04 07 093} ПЗ}\\[2em]
    \vfill

    \begin{tabular}{ p{0.65\textwidth}p{0.25\textwidth} }

      Студент: & Тропин~А.\,Г.\\
      \vfill & \vfill\\
      Руководитель & Сергиевский~М.\,В.\\
      \vfill
      Оценка\\
      \vfill
      Подпись членов комиссии
      % Консультанты: &\\
      % \hspace*{3ex}\emph{от кафедры информатики} & А.\,А.~Волосевич \\
      % \hspace*{3ex}\emph{по экономической части} & А.\,В.~Рябов \\
      % \hspace*{3ex}\emph{по охране труда} & Е.\,А.~Колосова \\
      & \\
    \end{tabular}
    \vfill
    {\normalsize Москва 2015}
  \end{center}
\end{titlepage}






  Разработка алгоритма для выделения частовстречающихся шаблонов ошибок из log-файлов программ
  \tableofcontents

  \chapter*{Введение}
  \addcontentsline{toc}{chapter}{Введение}
  Большинство программных систем, имеющих сложную структуру и состоящих из
нескольких сотен различных компонент, обладают рядом схожих проблем.
Например веб-поиск содержит следующие компоненты:
балансеры, верхние, средние метапоиски, промежуточные и базовые поиски,
колдунщики, антироботы, свежесть, региональные поиск,
несколько десятков параллельных поисков.
% Краткое схематичное описание архитектуры:

% \includegraphics[width=\textwidth]{pics/search.png}

\begin{definition}[Экземпляр]
  приложение, запущенное в контейнере и описываемое парой host:port.
\end{definition}
Всего единовременно запущено несколько ******** тысяч экземпляров приложений.
Каждый экземпляр генерирует множество ошибок и записывает каждую из них в
файл журнала ошибок.

Некоторые приложения, близкие по функционалу, пишут в один и тот же файл.
Файлы журналов ротируются согласно определённому алгоритму. Тем не менее объем
файла журнала для одного экземпляра может достигать нескольких сотен мегабайт,
что препятствует быстрому ручному анализу в случае инцидента и инженеры
вынуждены тратить ценные секунды на просмотр сотен тысяч строк файла в поисках
сообщения с описанием элемента, вызвавшего сбой работы системы.

Начальным требованием к системе для эффективного использования алгоритма
является наличие сопоставимого с количеством поисковых экземпляров
приложений количества узлов на которых может быть запущена программа,
реализующая алгоритм.

В организации, в которой выполнялась учебно-исследовательская работа,
развёрнута большая поисковая инфраструктура, которая не лишена недостатков и
существует вероятность поломки некоторой её части. Существует множество
средств мониторинга состояния веб-поиска и противодействия инцидентам,
но в некоторых случаях инженерам их недостаточно и приходится вручную
анализировать файлы журналов отдельных экземпляров приложений на отдельных
сервераз, что, в свою очередь, замедляет скорость реакции на непредвиденную
ситуацию. Но даже автоматизация процесса анализа файла журнала одного
экземпляра не решает проблему полностью, поэтому необходима возможность
быстрого анализа файлов журналов сразу множества экземпляров.

\begin{definition}[Шаблон]
  Специально подготовленное регулярное выбражение с экранированными
  спецсимволами.
\end{definition}
Таким образом, целью этой учебно-исследовательской работы является разработка
алгоритма, позволяющего собирать статистику по ошибкам, встречающимся
в файлах журналов экземпляров поисковых приложений на основе существующих
шаблонов и выделять новые шаблоны.

  \chapter{Анализ предметной области}
  \section{Представление задачи в терминах MapReduce}
Для распределённого запуска приложений была выбрана модель распределённых
вычислений MapReduce\footnotemark
\footnotetext{MapReduce — модель распределённых вычислений, представленная
  компанией Google, используемая для параллельных вычислений над очень
большими, несколько петабайт, наборами данных в компьютерных кластерах.}
по ряду причин.

Как оказалось задача без особых сложностей выражается в терминах
чистых функций flat map и flat reduce, так как основная структура,
используемая в алгоритме --- это пара(паттерн\footnotemark
\footnotetext{паттерн --- регулярное выражение},
количество совпадений) и благодаря этому однопоточный код легко запускается
на множестве нод MapReduce-кластера. Это обстоятельство освобождает
от реализации сложного механизма сетевого взаимодействия.

В качестве реализации модели MapReduce была выбрана реализация
Yandex Tables. Код программы, реализующий алгоритм легко переносится
на другие реализации данной модели распределённых вычислений.


\section{Выбор языка программирования и средств разработки}
В качестве среды разработки было решено использовать удалённую виртуальную
машину с конфигурацией и окружением идентичными конфигурации и окружению
MapReduce ноды. Были выбраны следующие программные продукты:

\begin{itemize}

\item В качестве операционной системы использовался дистрибутив GNU/Linux
  Ubuntu 12.04 LTS с модифицированным ядром и дополнительными пакетами.

\item Vim --- один из немногих несуществующих настоящих текстовых редакторов,
  обладающих массой встроенных возможностей и практически безграничным
  потенциалом к расширению за счёт встроенного интерпретируемого языка
  программирования VimL и поддержкой возможности написания расширений на
  таких языках, как python, ruby, perl.

\item Сохранение состояние рабочего окружения осуществлялось с помощью терминального
мультиплексора tmux, имеющего клиент-серверную архитектуру и
позволяющего отсоединятся от текущей сессии, оставляя её работать в фоновом
режиме с последующей возможностью переподключения.
tmux — свободная консольная утилита-мультиплексор,
предоставляющая пользователю доступ к нескольким терминалам в рамках
одного экрана. tmux может быть отключён от экрана: в этом случае он
продолжит исполняться в фоновом режиме; имеется возможность вновь
подключиться к tmux, находящемуся в фоне. tmux является штатным
мультиплексором терминалов операционной системы OpenBSD.
Программа tmux задумывалась как замена программы GNU Screen.

\item Удалённое подключение осуществлялось средствами защищённого протокола
ssh\\
(беспарольная аутентификация с использованием ключа) и технологии cauth.
SSH (англ. Secure Shell — ``безопасная оболочка'']) — сетевой протокол
прикладного уровня, позволяющий производить удалённое управление операционной
системой и туннелирование TCP-соединений (например, для передачи файлов).
Схож по функциональности с протоколами Telnet и rlogin, но, в отличие от них,
шифрует весь трафик, включая и передаваемые пароли. SSH допускает выбор
различных алгоритмов шифрования. SSH-клиенты и SSH-серверы доступны для
большинства сетевых операционных систем.

\item Для управления версиями исходных кодов использовалась децентрализованная
  система управления версиями git, в качестве hosting-провайдера был
  использован сервис github

\end{itemize}

В качестве языка программирования был выбран python2.7. Так как:

\begin{itemize}
\item Он предустановлен в большинстве современных дистрибутивах
  операционных систем.
\item Выразителен. Аналогичные программы на таких языках как Java, C++
  имеют в разы большие объёмы исходных кодов.
\item Обладает высокой производительностью.
\item Имеет множество библиотек, в том числе библиотеки для работы с
  регулярными выражениями, обёртки для MapReduce.

\end{itemize}

\section{Структура данных}
В качестве входных данных использовался агрегированный log-файл с нерегулярной
структурой, содержащий сообщения об ошибках в различных форматах, как
многострочные, так и однострочные.

Первая часть алгоритма, позволяет осуществить сбор статистики
и возвращает список пар (паттерн, количество совпадений), так же существует
возможность получить часть текста не удовлетворяющую известным паттернам,
для последующего анализа с помощью второй части алгоритма, позволяющей
выделить новые предполагаемые паттерны и с помощью первой части алгоритма
получить статистику, подтверждающую или опровергающую предположение.

\section{Стадии выполнения задачи}
\subsection{Подготовка репозитория}
Был создан git-репозиторий на сервисе github, сделана его локальная копия.
Была выбрана следующая структура проекта и правила именования файлов:

\begin{itemize}
\item В корневом каталоге лежат файлы README.md, LICENSE, .gitignore.
\item В каталоге doc/ хранится документация. Исходные тексты в \LaTeX и
  скомпилированная версия в PDF.
\item В каталоге direlog/ хранятся исходные коды с расширением .py и тесты,
  имеющие префикс test\_
\item В каталоге direlog/example/ хранятся примеры log-файлов и
  вспомогательные скрипты на языке bash.
\end{itemize}

\subsection{Выбор хранилища для паттернов}
В качестве хранилища для паттернов было решено использовать обычный файл на
языке python, названный patterns.py и содержащий в себе два списка паттернов
prepare\_patterns и main\_patterns, используемых на подготовительном этапе
и на этапе сбора статистики соответственно,
и хранить его под контролем версий в этом же репозитории. Причин
на это несколько: во-первых простота модификации файла с помощью скриптов,
во-вторых возможность просмотра истории изменений и откат к предыдущим версиям,
в-третьих возможность ручного редактирования.

\subsection{Написание программного кода}

Написание программного кода было разделено на несколько этапов.
\begin{enumerate}
\item Написание утилиты для предварительной обработки исходного log-файла.
  Утилита получила название prepare.py и позволила подготовить сырой log-файл
  для последующей обработки, путём замены уникальных токенов, таких
  как UUID, timestamp, версии, номера строк, пути, содержащие версии, на
  строковые константы.
\item Формирование prepare\_patterns на основе ручного анализа
  log-файлов.
\item Написание функциональных тестов для prepare.py.
\item Реализация алгоритма для сбора статистики с использованием
  main\_patterns. Утилита получила название direlog.py. И позволяет на
  выходе получить список пар (паттерн, количество совпадений) или текст,
  не подходящий под известные паттерны.
\item Добавление поддержки буферизации входного потока и поддержки
  многострочных паттернов.
\item Написание функциональных тестов для direlog.py.
\item Добавление функции, позволяющей запускать алгоритм на MapRedcue-кластере.
\end{enumerate}



  \chapter{Имплементация задачи}
  \section{Диаграмма компонентов}
  \section{Описание способов запуска}
  % Структурная схема приложения питонячие файлики, тесты, примеры логов.
  % как запускаются, как работают.
  \section{Анализ полученных результатов}
  % Что получил

  \chapter{Дальнейшее развитие}
  % Автоматическое выделение паттернов. Итерационная схема с эвристикой.
  % Интеграция с другими программными продуктами.



  \chapter*{Заключение}
  \addcontentsline{toc}{chapter}{Заключение}
  % Делали, сделали за пол года такой кусок, есть план на следующий год.

  % \part{}
  %\input{ch1}

\end{document}

