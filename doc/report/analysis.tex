\section{Представление задачи в терминах MapReduce}
Для распределённого запуска приложений была выбрана модель распределённых
вычислений MapReduce\footnotemark
\footnotetext{MapReduce — модель распределённых вычислений, представленная
  компанией Google, используемая для параллельных вычислений над очень
большими, несколько петабайт, наборами данных в компьютерных кластерах.}
по ряду причин.

Как оказалось задача без особых сложностей выражается в терминах
чистых функций flat map и flat reduce, так как основная структура,
используемая в алгоритме --- это пара(паттерн\footnotemark
\footnotetext{паттерн --- регулярное выражение},
количество совпадений) и благодаря этому однопоточный код легко запускается
на множестве нод MapReduce-кластера. Это обстоятельство освобождает
от реализации сложного механизма сетевого взаимодействия.

В качестве реализации модели MapReduce была выбрана реализация
Yet Another Map Reduce. Код программы, реализующий алгоритм легко переносится
на другие реализации данной модели распределённых вычислений.


\section{Выбор языка программирования и средств разработки}
В качестве среды разработки было решено использовать удалённую виртуальную
машину с конфигурацией и окружением идентичными конфигурации и окружению
MapReduce ноды. Были выбраны следующие программные продукты:

\begin{itemize}

\item В качестве операционной системы использовался дистрибутив GNU/Linux
  Ubuntu 12.04 LTS с модифицированным ядром и дополнительными пакетами.

\item Vim --- один из немногих несуществующих настоящих текстовых редакторов,
  обладающих массой встроенных возможностей и практически безграничным
  потенциалом к расширению за счёт встроенного интерпретируемого языка
  программирования VimL и поддержкой возможности написания расширений на
  таких языках, как python, ruby, perl.

\item Сохранение состояние рабочего окружения осуществлялось с помощью терминального
мультиплексора tmux\footnotemark, имеющего клиент-серверную архитектуру и
позволяющего отсоединятся от текущей сессии, оставляя её работать в фоновом
режиме с последующей возможностью переподключения.
\footnotetext{tmux — свободная консольная утилита-мультиплексор,
предоставляющая пользователю доступ к нескольким терминалам в рамках
одного экрана. tmux может быть отключён от экрана: в этом случае он
продолжит исполняться в фоновом режиме; имеется возможность вновь
подключиться к tmux, находящемуся в фоне. tmux является штатным
мультиплексором терминалов операционной системы OpenBSD.
Программа tmux задумывалась как замена программы GNU Screen.}

\item Удалённое подключение осуществлялось средствами защищённого протокола
ssh\footnotemark \\
(беспарольная аутентификация с использованием ключа) и технологии cauth.
\footnotetext{SSH (англ. Secure Shell — «безопасная оболочка»]) — сетевой протокол
прикладного уровня, позволяющий производить удалённое управление операционной
системой и туннелирование TCP-соединений (например, для передачи файлов).
Схож по функциональности с протоколами Telnet и rlogin, но, в отличие от них,
шифрует весь трафик, включая и передаваемые пароли. SSH допускает выбор
различных алгоритмов шифрования. SSH-клиенты и SSH-серверы доступны для
большинства сетевых операционных систем.}

\item Для управления версиями исходных кодов использовалась распределённая
  система управления версиями git, в качестве hosting-провайдера был
  использован сервис github

\end{itemize}


\section{Структура данных}
\section{Стадии выполнения задачи}
\subsection{Написание программного кода}
\subsection{Написание функциональных тестов}
