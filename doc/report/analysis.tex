\section{Представление задачи в терминах MapReduce}
Для распределённого запуска приложений была выбрана модель распределённых вычислений MapReduce \footnote{MapReduce — модель распределённых вычислений, представленная компанией Google, используемая для параллельных вычислений над очень большими, несколько петабайт, наборами данных в компьютерных кластерах.} по ряду причин.

Как оказалось задача без особых сложностей выражается в терминах чистых функций flat map и flat reduce, так как основная структура, используемая в алгоритме --- это пара(паттерн\footnote{паттерн --- регулярное выражение}, количество совпадений) и благодаря этому однопоточный код легко запускается на множестве нод MapReduce-кластера. Это обстоятельство освобождает от реализации сложного механизма сетевого взаимодействия.

В качестве реализации модели MapReduce была выбрана реализация Yet Another Map Reduce. Код программы, реализующий алгоритм легко переносится на другие реализации данной модели распределённых вычислений.


\section{Выбор языка программирования и средств разработки}
\section{Структура данных}
\section{Стадии выполнения задачи}
\subsection{Написание программного кода}
\subsection{Написание функциональных тестов}
