В процессе выполнения учебно-ислледовательской работы был разработан
алгоритм, позволяющий собирать статистику количества совпадений шаблонов
в файлах журналов.
Алгоритм был реализован на языке python с использование MapReduce-системы
Yandex Tables.

Кроме сбора статистики приложение позволяет получить контекст шаблона,
что полезно в множестве случаев при анализе файла журнала.

С помощью собранной статистики, приложение позволило продемонстритровать
проблемы существующей подсистемы отладочного вывода и форсировать её
рефакторинг.

После рефакторинга скорость роста файла журнала уменьшилась с полутора мегабайт
в секунду до одного мегабайта в секунду. В отладочные сообщения была добавлена
дополнительная информация.

Так же разработанное приложение позволяет исключить заведомо неинформативные
сообщения об ошибках, что в совокупности с предыдущим фактом уменьшает время
ручного анализа файла журнала инженером с двух минут до сорока секунд.

Ввиду сложности оценки корректности предположения, не было реализовано
автоматическое выделение шаблонов с помощью эвристики. Поэтому выделение новых
шаблонов осуществляется в ручном режиме.

Направление дальнейшего развития приложения --- автоматическое выделение
шаблонов средствами машинного обучения.
