\begin{definition}[MapReduce]
  Модель распределённых вычислений, представленная
  компанией Google, используемая для параллельных вычислений над очень
  большими, несколько петабайт, наборами данных в компьютерных кластерах.
\end{definition}

\begin{definition}[Map]
  map(f, list) --- функция высшего порядка двух аргументов, применяет к каждому
  элементу списка list, функцию f(x), в качестве результата возвращает список
  полученных значений.
\end{definition}

\begin{example}
  map(lambda x: x**3, [1, 2, 3]) вернёт список [1, 8, 27].
\end{example}

\begin{definition}[Reduce]
  reduce(f, list, init=None) --- функция высшего порядка, последовательно
  применяет f(x, y) к элементу списка и значению от предыдущего выполнения
  функции.
\end{definition}

\begin{example}
  reduce(lambda x, y: x + y, [1, 2, 3]) вернёт 6(сумма всех элементов).
  6~=~sum(sum(1,~2),~3).
\end{example}

\begin{definition}[Экземпляр]
  Приложение, запущенное в контейнере и описываемое парой host:port.
\end{definition}

\begin{definition}[Шаблон]
  Специально подготовленное регулярное выбражение с экранированными
  спецсимволами.
\end{definition}

\begin{definition}[Эвристический алгоритм (эвристика)]
  Алгоритм решения задачи, не имеющий строгого обоснования, но, тем не менее,
  дающий приемлемое решение задачи в большинстве практически значимых случаев.
\end{definition}

