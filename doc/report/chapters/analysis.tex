\section{Актуальность задачи}
Достаточно крупных систем, для которых поставленная задача могла бы иметь
актуальность, не так много и в основном это коммерческие проекты таких
компаний как: google, yahoo, yandex, amazon, facebook, twitter.
Откуда вытекает два обстоятельства:
\begin{enumerate}
  \item Продукты разрабатываемые компаниями не выкладываются в свободный доступ.
  \item Продукты разрабатываемые сторонними компаниями не учитывают специфику
    крупных распределённых систем.
\end{enumerate}
Ещё одним важным аспектом является тот факт, что существующие приложения
предполагают, что формат файла журнала заранее известен, но это бывает не
всегда так. Например, если продукт разрабатывается в течении 20 лет
исправление ошибки, заложенной в начале разработки, обходится очень дорого.
Поэтому на порядок дешевле написать алгоритм, который позволяет анализировать
неструктурированные файлы журналов, чем исправлять архитектурные ошибки.


\section{Сравнение с существующими аналогами}
Ниже представлен список существующих проприетарных и свободных решений,
выложенных в открытом доступе, с описанием их недостатков:
\begin{itemize}
  \item ascolog.
    \href{https://www.ascolog.com/content/about-ascolog}
    {https://www.ascolog.com/content/about-ascolog}

    \begin{itemize}
      \item Проприетарный. Отсутствие исходного кода создаёт множество проблем,
        таких как: отсутствие возможности расширения функционала,
        возможное наличие вредоносных компонентов.
      \item Необходимая операционная система для запуска: Windows. К сожалению
        это достаточно редкое решение для крупных распределённых систем.
        Критичный пункт.
      \item Нет возможности для горизонтального масштабирования.
        Вертикальное масштабирование сильно ограниченно по своим возможностям.
        Ещё один критичный пункт.
      \item Нет возможности для расшариения функционала, какими-либо
        встроенными средствами.
      \item Сайт производителя не обновлялся больше 6 месяцев и, по всей
        видимости, продукт больше не развивается.
    \end{itemize}

  \item sawmill.
    \href{https://www.sawmill.net/index.html}
    {https://www.sawmill.net/index.html}

    \begin{itemize}
      \item Проприетарный. Отсутствие исходного кода создаёт множество проблем,
        таких как: отсутствие возможности расширения функционала,
        возможное наличие вредоносных компонентов.
      \item Нет возможности для расшариения функционала, какими-либо
        встроенными средствами.
      \item Предполагается заранее известная структура логов. Критичный пункт.
    \end{itemize}
\end{itemize}

Из-за наличия критичных проблем каждого из вышеперечисленных решений,
они не подходят для решения имеющейся проблемы.

\section{Постановка задачи}

Необходимо разработать алгоритм, позволяющий:
\begin{itemize}
  \item Собирать статистику с использованием известных шаблонов.
  \item Выделять новые шаблоны с помощью эвристических методов.
  \item Производить лёгкое масштабирование программы, реализующей алгоритм
    с использованием существующих технологий.
  \item Посмотреть контекст сообщения об ошибке(несколько строк сверху
    и снизу).
\end{itemize}

Требование к программе, реализующей алгоритм:
\begin{itemize}
  \item Высокая производительность(файл, размеров в 100МБ должен
     обрабатываться не более 15 секунд).
  \item Возможность запуска под операционной системой семейства GNU/Linux.
  \item Горизонтальная масштабируемость.
\end{itemize}



